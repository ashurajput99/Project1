% Generated by GrindEQ Word-to-LaTeX 
\documentclass{article} % use \documentstyle for old LaTeX compilers

%\usepackage[english]{babel} % 'french', 'german', 'spanish', 'danish', etc.
\usepackage{amssymb}
\usepackage{amsmath}
%\usepackage{txfonts}
%\usepackage{mathdots}
%\usepackage[classicReIm]{kpfonts}
\usepackage{graphicx}

% You can include more LaTeX packages here 


\begin{document}

%\selectlanguage{english} % remove comment delimiter ('%') and select language if required


\noindent 

\noindent 

\noindent Institute Of Management Research And Development College Shirpur                                         Page {\textbar} 1

\noindent 

\noindent \textbf{                            }

\noindent \textbf{A}

\noindent \textbf{PROJECT REPORT}

\noindent \textbf{ON}

\noindent \textbf{     ``}\underbar{PAYROLL MANAGEMENT SYSTEM}\textbf{''}

\noindent \textbf{    At}

\noindent \textbf{         ``Flow Comptech Solutions LLP''}

\noindent 
\paragraph{Submitted By}

\noindent \textbf{Rajput Ashwini Jeevansing}

\noindent 
\subsection{                                                    Guided by}

\noindent \textbf{ Mr. Rahul chaudhari}

\noindent \includegraphics*[width=2.15in, height=1.90in, keepaspectratio=false, trim=0.00in 0.02in 0.04in 0.00in]{image1}

\noindent \textbf{Institute Of Management Research And Development Shirpur, }

\noindent \textbf{Dhule-425405}

\noindent \textbf{Department of Computer Application }

\noindent \textbf{Submitted to}

\noindent \textbf{K. B. C. North Maharashtra University, Jalgaon.}
\[2021-2022\] 
\textbf{}

\noindent \textbf{\textit{CERTIFICATE}}

\noindent  This is to certify that \textbf{Rajput Ashwini Jeevansing }student of \textbf{MCA} has completed Project Title as ``\textbf{\textit{PAYROLL MANAGEMENT SYSTEM}}'' \textbf{\textit{ }}'' At ``Flow Comptech Solutions LLP'' \textbf{ }under the guidance of \textbf{Mr. Rahul chaudhari} during the academic year 2021-2022.

\noindent \textbf{Date: }

\noindent \textbf{Place: Shirpur}

\noindent \textbf{}

\noindent \textbf{}

\noindent \textbf{           Guide                    H.O.D}

\noindent \textbf{    Mr.Rahul Chaudhari                                             Prof. Manoj Behere}

\noindent \textbf{}

\noindent \textbf{Examiners:}

\noindent \textbf{ 1)    External  : \_\_\_\_\_\_\_\_\_\_\_\_\_\_\_\_\_\_\_\_\_\_\_\_\_\_\_\_\_\_\_\_}

\noindent \textbf{ 2)    Internal   : \_\_\_\_\_\_\_\_\_\_\_\_\_\_\_\_\_\_\_\_\_\_\_\_\_\_\_\_\_\_\_\_       }

\noindent \textbf{         ACKNOWLEDGEMENT}

\noindent \textbf{}

\noindent It gives me immense pleasure for submitting this project report on ``\textbf{Payroll Management System'''' }for degree course on Bachelor of Computer Application.

\noindent I am thankful to my Guide\textbf{ Mr. Rahul Chaudhari,} for valuable guidance, kind of suggestion, constant encouragement and excellent co-operation and valuable help.

\noindent ~I wish to record my deep gratitude and sincere thanks to \textbf{Prof. Manoj Behere,} Head of Department, for their kind support, inspiration and timely providing required facilities for the completion of project.~

\noindent I also express my gratitude and thanks to \textbf{Prof. vaishali Patil,} Principal, R.C.P.A.C.S College Shirpur, for permitted me to carry out this Project.

\noindent  Last but not the least my sincere thanks to our parents, family members and friends for their continuous support, inspiration and encouragement without which this project would not have been success.

\noindent 

\noindent 

\noindent 

\noindent 

\noindent 

\noindent \textbf{Miss. Rajput Ashwini Jeevansing}

\noindent \textbf{                                                                                                                (MCA )}

\noindent \textbf{                                                                                                                       (2021-2022)}

\noindent \textbf{}

\noindent \textbf{\underbar{}}

\textbf{                                             }

\noindent \textbf{}

\noindent \textbf{\underbar{INDEX}}

\begin{tabular}{|p{0.8in}|p{3.0in}|p{0.8in}|} \hline 
\textbf{Chapter No} & \textbf{Name Of Chapter} & \textbf{Page No} \\ \hline 
\textbf{1} & \textbf{Introduction} & \textbf{1} \\ \hline 
\textbf{} & 1.1 Introduction to project &  \\ \hline 
\textbf{} & 1.2 Objective of project &  \\ \hline 
\textbf{} & 1.3 Scope of project &  \\ \hline 
\textbf{2} & \textbf{System Study And Analysis} & \textbf{} \\ \hline 
\textbf{} & 2.1 Information gathering & \textbf{} \\ \hline 
\textbf{} & 2.2 Existing System & \textbf{} \\ \hline 
\textbf{} & 2.3 Drawbacks in existing system & \textbf{} \\ \hline 
\textbf{3} & \textbf{Need Of Computer} & \textbf{} \\ \hline 
\textbf{} & 3.1 Need for system & \textbf{} \\ \hline 
\textbf{} & 3.2 need for computerization & \textbf{} \\ \hline 
\textbf{} & 3.3 Advantages of computerization & \textbf{} \\ \hline 
\textbf{} & 3.4 Introduction to proposed computerization system  & \textbf{} \\ \hline 
\textbf{} & 3.5 Advantages of computerized system & \textbf{} \\ \hline 
\textbf{} & 3.6 system security & \textbf{} \\ \hline 
\textbf{4} & \textbf{Feasibility Study} & \textbf{} \\ \hline 
\textbf{} & 4.1 Economical feasibility & \textbf{} \\ \hline 
\textbf{} & 4.2 Technical feasibility & \textbf{} \\ \hline 
\textbf{} & 4.3 Operational feasibility & \textbf{} \\ \hline 
\textbf{5} & \textbf{Hardware/software Requirement} & \textbf{} \\ \hline 
\textbf{6} & \textbf{Reason for selecting php} & \textbf{} \\ \hline 
\textbf{7} & \textbf{Preliminary Design for proposed system} & \textbf{} \\ \hline 
\textbf{} & 7.1 Data designing & \textbf{} \\ \hline 
\textbf{} & 7.2 Input designing & \textbf{} \\ \hline 
\textbf{} & 7.3 Output designing & \textbf{} \\ \hline 
\textbf{} & 7.4 Screen designing & \textbf{} \\ \hline 
\textbf{} & 7.5 Procedural designing & \textbf{} \\ \hline 
\textbf{} & 7.6 Architectural designing & \textbf{} \\ \hline 
\textbf{8} & \textbf{Detail design of proposed system} & \textbf{} \\ \hline 
\textbf{} & 8.1 Detail designing & \textbf{} \\ \hline 
\textbf{} & 8.2 data dictionary & \textbf{} \\ \hline 
\textbf{9} & \textbf{Appendix A} & \textbf{} \\ \hline 
\textbf{} & 9.1 Data entry screens as forms & \textbf{} \\ \hline 
\textbf{10} & \textbf{Appendix B} & \textbf{} \\ \hline 
\textbf{} & 10.1 Sample reports generated by computerized system & \textbf{} \\ \hline 
\textbf{11} & \textbf{Conclusion} & \textbf{} \\ \hline 
\textbf{12} & \textbf{References} & \textbf{} \\ \hline 
\end{tabular}

\textbf{\underbar{}}

\noindent \textbf{}

\noindent \textbf{}

\noindent \textbf{}

\noindent \textbf{}

\noindent \textbf{}

\noindent \textbf{}

\noindent \textbf{}

\noindent \textbf{}

\noindent \textbf{}

\noindent \textbf{}

\noindent \textbf{}

\noindent \textbf{}

\noindent \textbf{}

\noindent \textbf{}

\noindent \textbf{}

\noindent \textbf{}

\noindent \textbf{}

\noindent \textbf{}

\noindent \textbf{}

\noindent \textbf{}

\noindent \textbf{}

\noindent \textbf{}

\noindent \textbf{}

\noindent \textbf{   }

\noindent \textbf{}

\noindent \textbf{}

\noindent \textbf{}

\noindent \textbf{                               \underbar{Introduction}}\underbar{}

\noindent 

\noindent Institute Of Management Research and Development\textbf{ College (IMRD), Shirpur}

\noindent Was\textbf{ }established in 1997. \textbf{}

\noindent Its affiliation to North Maharashtra University. 

\noindent \textbf{Mission }

\noindent The mission of the college is to make their students responsible, sensitive, socially committed and to develop in them spiritual insight and the ideas of patriotism, democracy, secularism, socialism and peace. 

\noindent \textbf{Vision }

\noindent \textbf{Vision }

\noindent Their vision is to be a pre-eminent educational institute where teaching and learning bring out the best in the students and to inculcate in them the good qualities of heart and head which will help them to make the world around better place to live in. 

\noindent IMRD  college provides for instruction in various under-graduate and post-graduate courses in the faculties of Commerce and Science. 

\noindent \textbf{1.2 Objective of project: }

\noindent 

\noindent The main objective of this system is to manages all the information about Payments, Employees, Payroll. The project is totally built at administrative end and thus only the administrator is guaranteed the access.It helps company or organizations to maintain its Employees details, Department details, Staff Appoinment Details, Staff Relive etc. In other words, our Payroll Management System has, following objectives:

\noindent 

\begin{enumerate}
\item  Simple database is mantained

\item  Easy operations for the operator of the system. 

\item  To store up-to date information of the employees. 
\end{enumerate}

\noindent 

\noindent \textbf{\underbar{1.3 Scope of project: }}\underbar{}

\noindent 

\noindent This is generic type of software, suitable for all payroll system. It has separate divisions to handle the payment transactions. Separate division is provided to maintain teacher records, department  Records, Salary Record.

\noindent 

\noindent 

\noindent 

\noindent 

\noindent 

\noindent                                                        

\noindent                                                   Chapter No: - 2 

\noindent \textbf{                     \underbar{System Study and Analysis }}

\noindent 

\noindent \textbf{\underbar{2.1 Information gathering: }}

\noindent 

\noindent Investigation of system is first step while designing a system. This is the way to handle user's needs. 

\noindent We consider following things:- 

\noindent 1. How the precise system works? 

\noindent 2. Time taken to process the data through system. 

\noindent 3. List of documents. 

\noindent 4. Files, reports associated with system. 

\noindent 

\noindent From the available document we got basic idea about fundamental of system as well as input and output of this system. In next step we have about existing system and collected information about inputs, reports, transaction. 

\noindent \textbf{\underbar{2.2 Existing system: }}\underbar{}

\noindent 

\noindent Before developing the computerized system the firm managed the work manually in different modes i.e. work was divided into various steps. 

\noindent These steps were as follow:- 

\noindent 1. Making salary slip manually for a teachers. 

\noindent 2. Making register and maintain it manually. 

\noindent \underbar{}

\noindent \textbf{\underbar{2.3 Drawbacks of existing system: }}\underbar{}

\noindent 

\noindent 1. Accuracy is less because all work in done manually. 

\noindent 2. More time is required for data entry its does not avoid repetition of data entry. 

\noindent 3. More time is required to process required information in form of output ,hence reports required for management getting developed 

\noindent 4. Human mistakes may be possible. 

\noindent 5. Handling of all records and different files is tedious and lengthy jobs. 

\noindent 

\noindent 

\noindent 

\noindent                                                            Chapter No: - 3 

\noindent \textbf{                                   \underbar{Need of Computer }}

\noindent \underbar{}

\noindent \textbf{3.1 \underbar{Need of system: }}\underbar{}

\noindent \textbf{\underbar{3.2 Need for computerization: }}\underbar{}

\noindent 

\noindent The main aim of the computerizing any system is to provide less time, less Work, to increase efficiency and according so that large task can be done in short period of time. 

\noindent The computer is use to assist the man in business organization carry out large and wide variety Of activity .Accurate recording and processing of this activity are called as data processing .As the complexity and size of the organization is growing day by day, some degree this automation becomes necessary. 

\noindent When organization grows the manual system begin to break down, thus harming task of planning and controlling .In most of the organization electrical cost low and increase efficiency was introduce for processing of various transaction that arrives in difficult organization operation. 

\noindent Most organization carry out largest wide variety of business transaction, accurate recording and processing of this transaction is now as data processing .Today some degree of automation exists in all organization in degree and processing of daily transaction. 

\noindent \textbf{3.3 \underbar{Advantages of computerization:} }

\noindent 

\noindent Business organization found that they could computerize the transaction processing tasks quickly by using standard software. 

\noindent The major advantages of such software as follows:

\noindent 

\begin{enumerate}
\item  Simple handling and retrieving of data is very easy. 

\item  Easy for any type of search record. 
\end{enumerate}

\noindent ???All calculation is done externally so calculation solved within a second. 

\noindent ???Data entry time is negligible as compared to manual system while entering         customer's entries previous data will be reduced. 

\noindent 

\noindent 

\noindent \textbf{3.4 \underbar{Introduction to proposed System: }}\underbar{}

\noindent Our computerized system is specially design for Payroll Management System which is made exactly as per the need of the organization .it is designed to speed up activity in the existing system and to minimized the human error ,man power and optimized its cost with better accuracy. 

\noindent 

\noindent The system is made very simple to understand and operate. 

\noindent This system supports the 

\noindent Graphical User Interface because this is menu driven and every time short keys are used to works through the keyboard also. 

\noindent The system takes the data of Classes, Salary, Leave, College, Attendance etc .the system generate reports as follows: 

\begin{enumerate}
\item \begin{enumerate}
\item  Teacher Registration Report 

\item  Pay Slip Report 
\end{enumerate}
\end{enumerate}

\noindent 

\noindent ? In our proposed system we have the provision for adding the details of the payroll management. Another advantage of the system is that it is very easy to edit the details of the employees and delete a employees when it found unnecessary. Here is no facility of net connection, email facility is also not provided. Online payment is not possible. 

\noindent 

\noindent By developing the system, we can attain the following facilities:

\noindent 

\noindent ?Easy to handle and feasible. ??

\noindent ? Easy to operate. Cost reduction. 

\noindent 

\begin{enumerate}
\item \begin{enumerate}
\item  \textbf{\underbar{Advantages of computerized system: }}\underbar{}
\end{enumerate}
\end{enumerate}

\noindent 

\begin{enumerate}
\item  Data entry time is negligible as compare to manual system because while entering entries previous data will appeared so that required to complete entry will be reduced. 
\end{enumerate}

\noindent 

\begin{enumerate}
\item  The printed outputs are very neat, precise and even print attractively by simple pressing and clicking print button or report. 
\end{enumerate}

\noindent 

\begin{enumerate}
\item  Getting report or output is in time, hence it increase the productivity and popularity of center. 

\item  Handling and retrieving data is very easy. 
\end{enumerate}

\noindent 

\noindent \textbf{3.6 \underbar{System security:} }

\noindent Our system is security conscious and can be used by that person who has an authority to access the part of system software. 

\noindent We have created one type of login authority. Operators can entry and free to access part of system but he is responsible person so his responsibility increase in using system. In this way ensuring that not anybody can come and have access to our work. So system becomes more secure which is also an advantage of over manual system.

\noindent 

\noindent 

\noindent 

\noindent 

\noindent 

\noindent ??????????????\textbf{Chapter No: - 4 }

\noindent \textbf{                                   \underbar{Feasibility Study }}\underbar{}

\noindent \textbf{\underbar{Feasibility study: }}\underbar{}

\noindent          At the end of information gathering phase, we have data available currently and the deficiencies of the existing system. We also come to know the requirements. It is necessary to quantify goals and sub goals. 

\noindent                  Once this goals and sub goals are quantified, the next step is to find out whether these goals are met? And if yes, then how will they met? And at what cost? Feasibility analysis is mainly concern with these questions step followed in feasibility. 

\noindent                 All the projects are feasible given unlimited resources and infinite time. In the development of present system, no such restrictions for limitation are imposed that are not feasible. As all the resources are easily available and the given was enough. 

\noindent 

\noindent     Feasibility study can be divided into 3 broad areas:

\noindent 

\noindent 1.Economic feasibility 

\noindent 2. Technical feasibility 

\noindent 3. Operational feasibility 

\noindent 

\noindent \textbf{4.1 \underbar{Economical feasibility study: }}\underbar{}

\noindent Any system before implementation must be checked whether it is feasible or not. 

\noindent        I.e. the new system is more feasible than existing system. 

\noindent 

\noindent If proposed system is implemented then there are following benefits: 

\noindent 

\noindent 1. The man power required for maintenance of system is reduced. 

\noindent 2. Faster and accurate processing of transactions. 

\noindent 3. Easier and understanding of working system by the operator. 

\noindent 4. Easier and instant report generation. 

\noindent 

\noindent         The organization has implemented the local area networking operating system with the dedicated file servers. Therefore the implementation of multi-user system is easy as the operating system level.

\noindent 

\noindent \textbf{4.2 \underbar{Technical feasibility study:} }

\noindent              During technical feasibility analysis, the analyst evaluates the technical merits of the system concept, while at the same time collects additional information about performance, reliability and predictability. 

\noindent Technical feasibility analysis begins with an assessment of the technical viability of proposed system. It is analyzed that what kind of development environment is required, what new method and processor are required to accomplish systems function and performance. A model is created based on the systems goal. A solution is technically feasible, if technology is available to implement it as we provided with sufficient support, we do not have a technical feasibility problem.

\noindent 

\noindent \textbf{4.3 \underbar{Operational feasibility study:} }

\noindent 1. As top management have full support for project due to higher usefulness and advantages of system. 

\noindent 

\noindent 2. Also user support is available is because the way which existing system work causes many drawbacks. 

\noindent 

\noindent 3. Also current methods are not acceptable to user due to its time consuming nature. 

\noindent 

\noindent 4. User is already in the phase of planning to bring change in the current system. So the project is operationally most feasible and no resistance from users. 

\noindent 

\noindent 5. Proposed system causes no any harm to anyone. But on other hand it has much vital application, which causes benefits as ever. 

\noindent 

\noindent 6. Also proposed system performance is much quicker and faster than existing one and it has no effect on functionality of similar areas. 

\noindent 

\noindent 7. More important point is that system gives all types inventory status report.

\noindent 

\noindent 

\noindent 

\noindent 

\noindent 

\noindent 

\noindent 

\noindent 

\noindent 

\noindent 

\noindent 

\noindent 

\noindent 

\noindent \textbf{                                                CHAPTER NO: - 5 }

\noindent \textbf{                            \underbar{Hardware and Software }}\underbar{}

\noindent \textbf{\underbar{Hardware: }}\underbar{}

\noindent          Hardware of computer means all physical parts contained in the computer system. Following are the hardware that human use for performing various tasks: 

\noindent 1. Input devices. 

\noindent 2. Output devices. 

\noindent 3. Central processing unit. 

\noindent \textbf{\underbar{Configuration of hardware:}}\underbar{}

\begin{tabular}{|p{1.6in}|p{1.8in}|} \hline 
\textbf{ PROCESSOR: } & Intel(R) Core $\mathrm{{}^{TM}}$ i3 -4170T CPU @ 3.20GHz  \\ \hline 
\textbf{RAM: } & 4.00 GB  \\ \hline 
\textbf{HARD DISK: } & 500 GB  \\ \hline 
\end{tabular}

\textbf{}

\noindent \textbf{\underbar{Software:}}\underbar{}

\begin{tabular}{|p{1.1in}|p{1.1in}|} \hline 
\textbf{OPERATING SYSTEM: } & Windows-8.1  \\ \hline 
\textbf{FRONT END: } & Visual Studio 2012  \\ \hline 
\textbf{BACK END: } & Microsoft Access 2010  \\ \hline 
\end{tabular}



\noindent 

\noindent 

\noindent 

\noindent 

\noindent 

\noindent 

\noindent 

\noindent 

\noindent 

\noindent 

\noindent 

\noindent 

\noindent 

\noindent 

\noindent 

\noindent 

\noindent                                                            Chapter No: -6 

\noindent \textbf{                          \underbar{Reason for Selecting C\#.Net}}\underbar{}

\noindent \textbf{\underbar{What is .NET? }}\underbar{}

\noindent             .NET is essential framework of software development. It is similar in nature to any other software development framework (J2EE etc. in that it provides set of runtime containers/capabilities, and a rich set of pre-built functionality in the form of class libraries and API's.

\noindent \textbf{\underbar{.NET framework: }}\underbar{}

\noindent          The best way to understand what .NET offers is to observe some of the limitations of its predecessors. In this section, we take a brief and simplified look at the history of Microsoft component interaction and then a short look at the architecture. 

\noindent 

\noindent \textbf{\underbar{.NET architecture: }}\underbar{}

\noindent The .NET architecture consists of three parts: The Common Language Runtime, the framework classes and ASP.NET, which are covered in the following sections. The components of .NET tend to cause confusion.

\noindent 

\noindent \textbf{\underbar{Advantages of C\#.NET: }}\underbar{}

\begin{enumerate}
\item \underbar{ }C\# borrows concepts from Java and C++, adopting only the good bits from those languages and eliminating overly confusing and error prone features, which are the major sources of bugs in a code. 

\item  C\# is a terse language. It's very tiny even with the commands. Visual Basic on the other hand has a command for almost any kind of situation that the developer may face during the development of the code making its reference a real hefty one. 

\item  C\# supports effective and reusable components. 

\item  C\# is portable at the same time it is cross language compatible for all Microsoft Windows based languages and programs specifically targeted to that particular platform can be coded to interoperate with the code of other languages. 
\end{enumerate}

\noindent 

\noindent 

\begin{enumerate}
\item  C\# implements the modern programming concept of Object Oriented Programming which enables the developer to produce secure data centric applications and take the user to the next level of experience. 

\item  C\# programs can be written in as simple as a text pad and a command line which are common to any operating system provided the developer has installed the CLR and the framework priory. Microsoft's Rapid Application Development Suite products, named Microsoft Visual Studio ships with a separate Visual tool for C\#, and gives developers visually rich tools for development and deployment. 

\item  C\# RAD tools gives the developer the power to produce ``One click install'' application, where the user needs no prior software experience and can install and use C\# applications like any other windows program. 

\item  C\# provides the ability of code extension to the developer with which developers can produce extensions and wrappers to use the underlying library to behave the way the developer want it to. 

\item  C\# programs are managed code, to say, they are coded and executed in a controlled environment leaving little room for anomalies called ``bugs'' to creep in. Also it has eliminated some of the ``unsafe'' features of C++, which can provide intruders to breach secure C\# programs. 

\item  C\# can be used to write wide range of applications due to their portability, from simple desktop widgets to high end web services, secure systems programming and even robotics. 
\end{enumerate}

\noindent 

\noindent \textbf{\underbar{Features of .NET: }}\underbar{}

\noindent 

\noindent        1 Building Window Based Application. 

\noindent        2 Building Web-Based Application. 

\noindent        3 Building Console Based Application. 

\noindent        4 Simplified deployment. 

\noindent        5 Direct Access to Platform. 

\noindent        6 Full object oriented constructs. 

\noindent        7 XML Web services. 

\noindent         8 Mobile Application. 

\noindent 

\noindent \textbf{\underbar{Introduction to MS Access 2010: }}\underbar{}

\noindent         Microsoft Access is a part of the Microsoft Office Suite. Microsoft Office Access, previously known as Microsoft Access, is a database management system from Microsoft that combines the relational Microsoft Jet Database Engine with a graphical user interface.

\noindent 

\noindent 

\noindent 

\noindent 

\noindent 

\noindent 

\noindent \textbf{                                                 CHAPTER NO: - 7 }

\noindent \textbf{                \underbar{Preliminary designed of proposed system}}

\noindent 

\begin{enumerate}
\item  Data designing 

\item  Input designing 

\item  Output designing 

\item  Screen designing 

\item  Procedural designing 

\item  Architectural designing 
\end{enumerate}

\noindent 

\noindent \textbf{\underbar{Name of Table: Login Table}}

\noindent \textbf{\underbar{}}

\noindent \includegraphics*[width=3.16in, height=1.41in, keepaspectratio=false, trim=1.50in 5.84in 5.16in 1.64in]{image2}\underbar{}

\noindent \underbar{}

\noindent \textbf{\underbar{Name of Table: Teacher Registration Table}}

\noindent \textbf{\underbar{}}

\noindent \includegraphics*[width=3.56in, height=1.97in, keepaspectratio=false, trim=1.94in 4.20in 7.30in 1.35in]{image3}\underbar{}

\noindent \underbar{}

\noindent \textbf{\underbar{Name of Table: Generate Slip Table}}

\noindent \textbf{\underbar{}}

\noindent \includegraphics*[width=3.69in, height=1.75in, keepaspectratio=false, trim=2.10in 4.11in 7.59in 1.34in]{image4}\underbar{}

\noindent \underbar{}

\noindent \textbf{\underbar{Preliminary design of proposed system:}}

\noindent \textbf{\underbar{ }}\underbar{}

\noindent Preliminary design related with transformation of requirement into data. 

\noindent Data design defines data structure. (TABLES) 

\noindent \textbf{\underbar{7.1 Data design: }}

\noindent \underbar{}

\noindent Preliminary activity during data design is to select logical representation of data identifying during the requirement delimitation and specifications phase well data lead to program structure modularity and reduced procedural complexity. 

\noindent Designing is an interactive process of taking a logical model of system together with a strongly stated set of objective for that system and producing the specification of physical system that will meet these objectives.

\noindent \textbf{7.2 \underbar{Input Design:} }

\noindent 

\noindent         Though output are main determinant of any system performance the quality of output is determined by the input made to system carefully accepted data will give accurate and power analysis. Input having five types as external input, internal input, operation input, computerized input and interactive input. Our system checks the student data, save student data, modify or delete the student, teacher, class, subject, class, fees information. 

\noindent \textbf{7.3 \underbar{Output Design: }}

\noindent 

\noindent          The main objective of our system is to produce data to support operation or decision making of the organization. The output of our system is the primary contact between the system and user, our output are flexible. 

\noindent        Our system support query reports, summery reports, detail reports and period reports.

\noindent 

\noindent \textbf{\underbar{7.4 Screen Design: }}

\noindent 

\noindent Designing of the screen for the system have been designed with a throughout keeping in mind that it is very user easy to operate and error free, for data entry our system used interactive screen. We design menu path without breaking the integrity of security constraint. 

\noindent      Our system gives data entry form as 

\noindent ??Teacher data entry form: All the data related to every teacher is entered here. 

\noindent ??Teacher data entry form: All the data related to teacher salary is entered here. 

\noindent 

\noindent \textbf{\underbar{7.5 Procedural design: }}

\noindent \underbar{}

\noindent In procedural design occurred data and program structure in English. Procedural design requires defining algorithmic details of the procedures, which can be represent in natural language like English. Graphical tools such as flowchart or block diagram which provides excellent pictorial patterns back readily defines procedural design. 

\noindent           Flowchart is most particular graphical representation of procedural design. Architectural design is to develop modular program structure and represent the control relationships between modules. 

\noindent          Architectural design holds the program structure and data structure and defines interface enabling data to flow throughout the program.

\noindent 

\noindent \textbf{\underbar{}}

\noindent \textbf{\underbar{\eject }}

\noindent \textbf{\underbar{Data Flow Diagram: }}\underbar{}

\noindent A graphic tool used to describe and analyze the movement of data through a system manual or automated including the processes, store of data, and delays in the system. Data flow diagrams are the central tool and the basis from which other components are developed. The transformation of 

\noindent 

\noindent Data from input to output, through processes, may be described logically and independently of the physical components associated with the system. They are termed logical data flow diagrams in contrast, physical data flow diagram show the actual implementation and the movement of data. 

\noindent               A Data flow diagram is also known as `Bubble Chart' has the purpose of clarifying system requirements and identifying major transactions that will become programs in system 

\noindent              design. DFD describes the flow of data other than how they are processed. The data flow diagram generally contains the following elements and their representation is given below. 

\begin{enumerate}
\item  External Entity 

\item  Process 

\item  Data Flow 
\end{enumerate}

\noindent 

\noindent A level 0 DFD, also called a fundamental system model or a context model, represents the entire software element as a single bubble with input and output data indicated by incoming and outgoing arrows, respectively. Additional processes (bubbles) and information flow paths are represented as the level 0 DFD is partitioned to reveal more detail. For example, a level 1 DFD might contain five or six bubbles with interconnecting arrows. Each of the processes represented at level 1 is a sub-function of the overall system depicted in the context model. 

\noindent           Each of the bubbles may be refined or layered to depict more detail. Some graphical symbols used: 

\noindent 

\begin{enumerate}
\item  Process   
\end{enumerate}

\noindent                                           

\begin{enumerate}
\item  External Entity                            
\end{enumerate}

\noindent 

\noindent 

\begin{enumerate}
\item  Data Store                         
\end{enumerate}

\noindent 

\noindent       IV)     Data Flow 

\noindent 

\noindent ERD:

\noindent \includegraphics*[width=5.51in, height=4.41in, keepaspectratio=false, trim=3.21in 0.68in 3.52in 1.79in]{image5}

\noindent 

\noindent \underbar{Data Flow Diagrams:}

\noindent \underbar{0-level DFD:}

\noindent \includegraphics*[width=5.95in, height=4.54in, keepaspectratio=false, trim=1.57in 0.78in 1.74in 1.96in]{image6}\underbar{}

\noindent \underbar{}

\noindent \underbar{1-level DFD:-}

\noindent \includegraphics*[width=5.88in, height=4.17in, keepaspectratio=false, trim=1.54in 0.75in 2.12in 1.89in]{image7}\underbar{}

\noindent \underbar{}

\noindent \underbar{}

\noindent 

\noindent \textbf{                                                       }

\noindent \textbf{}

\noindent \textbf{}

\noindent \textbf{}

\noindent \textbf{}

\noindent \textbf{}

\noindent \textbf{}

\noindent \textbf{}

\noindent \textbf{}

\noindent \textbf{}

\noindent \textbf{}

\noindent \textbf{}

\noindent \textbf{}

\noindent \textbf{}

\noindent \textbf{}

\noindent \textbf{}

\noindent \textbf{}

\noindent \textbf{}

\noindent \textbf{}

\noindent \textbf{                                                              \underbar{Chapter:- 8 }}\underbar{}

\noindent \textbf{                    \underbar{Detail designed of proposed system}}\underbar{}

\noindent 

\noindent \textbf{\underbar{Detail design: }}\underbar{}

\noindent Specifying the algorithm and data structure that make up the interior modules does detail design of system. Usually there are many choices but from the different available alternatives. The one of which offers greatest, simply, functionality, and availability is selected based of relative importance of this criteria. 

\noindent      Detail design is converted into codes. Code optimization will improve the performance of the system.

\noindent \textbf{\underbar{Data dictionary: }}

\noindent 

\noindent Entire in data dictionary includes the names of data item and attributes. 

\noindent Data dictionary has been proposed a formal grammar for describing. 

\noindent The content of the definitions of all data is mentioned in data flow diagram. In process specification, composite data is define in terms of the meaning each of values that it can be assumed.

\noindent 

\noindent 

\noindent 

\noindent 

\noindent 

\noindent 

\noindent 

\noindent 

\noindent 

\noindent 

\noindent 

\noindent 

\noindent 

\noindent 

\noindent 

\noindent 

\noindent 

\noindent \textbf{                                                     \underbar{Chapter No: - 9 }}\underbar{}

\noindent \textbf{         }                                      \textbf{\underbar{Appendix A}}?\textbf{}

\noindent 

\noindent            Login form

\noindent             Main Form \textbf{         }

\begin{enumerate}
\item \textbf{ }Teacher Registration Form 

\item  Teacher Registration Detail Form 

\item  Generate Slip Detail Form 

\item  Pay Slip Detail Form 

\item  Change Password Form 

\item  

\item  \textbf{1. \underbar{Login form:-} }
\end{enumerate}

\noindent \includegraphics*[width=6.16in, height=5.28in, keepaspectratio=false]{image8}

\noindent 

\noindent 

\noindent \textbf{\underbar{2. Main form:- }}\underbar{}

\noindent 

\noindent 

\noindent \underbar{\includegraphics*[width=6.48in, height=4.89in, keepaspectratio=false]{image9}}

\noindent 

\noindent \textbf{\underbar{3. Teacher Registration form:- }}

\noindent \underbar{\includegraphics*[width=6.46in, height=6.57in, keepaspectratio=false]{image10}}

\noindent \textbf{\underbar{4.Teacher Registration Detail Form-}}

\noindent \underbar{\includegraphics*[width=6.52in, height=5.67in, keepaspectratio=false]{image11}}

\noindent \textbf{\underbar{5.Generate Slip Detail Form:-}}

\noindent \underbar{\includegraphics*[width=6.47in, height=5.99in, keepaspectratio=false]{image12}}

\noindent \textbf{\underbar{6.Pay Slip Detail Form:-}}

\noindent \underbar{\includegraphics*[width=6.53in, height=4.15in, keepaspectratio=false]{image13}}

\noindent \textbf{\underbar{7.Change Passward Form:-}}

\noindent \underbar{\includegraphics*[width=5.32in, height=5.06in, keepaspectratio=false]{image14}}

\noindent \underbar{}

\noindent 

\noindent \underbar{}

\noindent 

\noindent 

\noindent 

\noindent 

\noindent 

\noindent 

\noindent 

\noindent 

\noindent 

\noindent 

\noindent \textbf{                                                  Chapter No: - 10 }

\noindent \textbf{                                     \underbar{Appendix} \underbar{B}}

\noindent 

\noindent \textbf{\underbar{1. Teacher Registration Report }}\underbar{}

\noindent \textbf{\underbar{2. Pay Slip Report }}

\noindent \underbar{}

\noindent \underbar{}

\noindent 

\noindent 

\noindent 

\noindent 

\noindent 

\noindent 

\noindent 

\noindent 

\noindent 

\noindent 

\noindent 

\noindent 

\noindent 

\noindent 

\noindent 

\noindent 

\noindent 

\noindent 

\noindent 

\noindent 

\noindent 

\noindent 

\noindent \textbf{                                                    \underbar{Chapter No: - 11 }}\underbar{}

\noindent \textbf{                              \underbar{ Conclusion}}\underbar{}

\noindent Software qualities are reliability which includes all the new and advance facilities given by the visual basic. Software includes maintainability, modularity and good documentation making a system hundred percent reliable is nearly an impossible task, because through the testing who's resulted in great accuracy, the online nature may create on recover which may make system failure still ``Record maintaining'' has been developed to gain reliability it its maximum extends by considering all possible kinds of problem that may occur and create error in existing system. 

\noindent               By using this system, we maintain records of Teacher information and salary slip also maintain by this system. We can generate reports of all the forms and display the information. We can find the record of Teacher Information within minimum time. As per software quality is concerned, I tried to make the system user friendly and easy to understand.

\noindent 

\noindent 

\noindent 

\noindent 

\noindent 

\noindent 

\noindent 

\noindent 

\noindent 

\noindent 

\noindent 

\noindent 

\noindent 

\noindent 

\noindent 

\noindent 

\noindent 

\noindent \textbf{                               }

\noindent                                                               Chapter No: - 13 

\noindent \textbf{                                        \underbar{ References }}\underbar{}

\noindent \textbf{\underbar{13.1 References Book :- }}

\noindent 

\begin{enumerate}
\item  Mastering C\# - By E . Petreretsos . 

\item  Analysis and design of information system-By Jams A . Sem. 

\item  Access Computer References -- By Gorge Koch \& Kelvin Loney. 

\item  MS Office 2010 (BPB Publication) 
\end{enumerate}

\noindent \underbar{}

\noindent \underbar{}

\noindent \textbf{\underbar{13.2 web References:- }}\underbar{}

\noindent \textbf{\underbar{http://en.wikipedia.org/wiki/c-sharp.Net }}\underbar{}

\noindent \textbf{\underbar{http://msdn.micriosoft.com.uk/net/c\#.html }}\underbar{}

\noindent \textbf{\underbar{http://www.homeandlearn.co.uk/net/c\#.html }}\underbar{}

\noindent \textbf{\underbar{http://www.thedevelopertips.com./c\#.html}}\underbar{}

\noindent \underbar{}

\noindent 


\end{document}

